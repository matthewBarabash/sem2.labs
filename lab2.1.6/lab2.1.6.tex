\documentclass[a4paper,12pt]{article} % добавить leqno в [] для нумерации слева

%%% Работа с русским языком
\usepackage{cmap}					% поиск в PDF
\usepackage{mathtext}				% русские буквы в фомулах
\usepackage[T2A]{fontenc}			% кодировка
\usepackage[utf8]{inputenc}			% кодировка исходного текста
\usepackage[english,russian]{babel}	% локализация и переносы
\usepackage[left=2cm,right=2cm,top=2cm,bottom=2cm,footskip=1cm,includefoot]
{geometry}
%%% Дополнительная работа с математикой
\usepackage{amsmath,amsfonts,amssymb,amsthm,mathtools} % AMS
\usepackage{icomma} % "Умная" запятая: $0,2$ --- число, $0, 2$ --- перечисление
%\usepackage{textcomp}
%% Номера формул
\mathtoolsset{showonlyrefs=true} % Показывать номера только у тех формул, на которые есть \eqref{} в тексте.

%% Перенос знаков в формулах (по Львовскому)
\newcommand*{\hm}[1]{#1\nobreak\discretionary{}
	{\hbox{$\mathsurround=0pt #1$}}{}}

%%% Работа с картинками
\usepackage{graphicx}  % Для вставки рисунков
\graphicspath{{images/}{images2/}}  % папки с картинками
\setlength\fboxsep{3pt} % Отступ рамки \fbox{} от рисунка
\setlength\fboxrule{1pt} % Толщина линий рамки \fbox{}
\usepackage{wrapfig} % Обтекание рисунков и таблиц текстом
%%% Работа с таблицами
\usepackage{array,tabularx,tabulary,booktabs} % Дополнительная работа с таблицами

\usepackage{longtable}  % Длинные таблицы
\usepackage{multirow} % Слияние строк в таблице
\newcommand{\RomanNumeralCaps}[1]
{\MakeUppercase{\romannumeral #1}}

\usepackage{pgfplots, pgfplotstable}
\pgfplotsset{compat=1.9}

\usepackage{circuitikz}

\begin{document}	
	
\begin{minipage}[l]{0.3\textwidth}
	\textit{Работу выполнил}\\
	Просвирин Кирилл, 712гр.\\\\
	\textit{под руководством}\\
	А.В. Гаврикова, к.ф.-м.н.
\end{minipage}
\hfill
\begin{minipage}[l]{0.2\textwidth}
	Маршрут \RomanNumeralCaps{9} \\\\
	19 февраля 2018~г.,\\
	26 февраля 2018~г.\\
\end{minipage}
\\[20pt]
\begin{center}
	\LARGE{\textbf{Лабораторная работа № 2.1.6}\\
		Эффект Джоуля-Томсона \\[20pt]}
\end{center}

\textbf{Цель работы: }определение изменения температуры углекислого газа
при протекании через малопроницаемую перегородку при различных начальных
значенях давления и температуры; вычисление по результатам опытов
коэффициентов Ван-дер-Ваальса.\\

\textbf{В работе используется: }Термостат, дифференциальная термопара,
микровольтметр, манометр, установка (баллон, труба и т.д.).


\section{Теоретическая справка}

Эффект Джоуля–Томсона — изменение температуры газа, медленно протекающего
из области высокого в область низкого давления в условиях хорошей тепловой
изоляции.\\

\noindent В условиях данной лабораторной работы газ проходит через пористую
перегородку из области с давлением $ P1 $ в область с атмосферным давлением 
$ P2 $. Рассмотрим стационарный поток между сечениями \RomanNumeralCaps{1}
и \RomanNumeralCaps{2} до и после перегородки. Считая стенки
адиабатическими и жёсткими, из закона сохранения энергии получим: 
\begin{equation}\label{}
	A_1-A_2=
	\left(U_2+\dfrac{\mu v_2^2}{2}\right)-
	\left(U_1+\dfrac{\mu v_1^2}{2}\right),
\end{equation}
где $ A_1=P_1V_1 $~--- работа совершенная над газом при прохождении
через \RomanNumeralCaps{1}, $ A_2=P_2V_2 $~--- работа совершенная над
газом при прохождении через \RomanNumeralCaps{2}. Тогда
\begin{equation}\label{}
	H_1-H_2=\dfrac{1}{2}\mu(v_2^2-v_1^2).
\end{equation}
Поскольку скорости по обе стороны от перегородки малы, энтальпию можно
считать неизменной. Тогда 
\begin{equation}\label{}
	\mu_{дт}=\dfrac{\Delta T}{\Delta P}=
	-\dfrac{(\partial H/\partial P)_T}{(\partial H/\partial T)_P}.
\end{equation}
Нетрудно показать, что для идеального газа коэффициент Джоуля–Томсона равен
нулю, а для газа Ван-дер-Ваальса 
\begin{equation}\label{}
	\mu_{дт}=\dfrac{\frac{2a}{RT}-b}{C_P}.
\end{equation}
Температура, при которой $ \mu_{дт} $ меняет знак, называют температурой
инверсии: 
\begin{equation}\label{}
	T_{инв}=\dfrac{2a}{Rb}
\end{equation}


\section{Проведение измерений}
\begin{enumerate}
	\item Установим на термостате температуру $ T=24^\circ С $. 
	\item Откроем регулирующий вентиль и запишем показания 
	манометра $ \Delta P $~и~вольтметра~$ U $. 
	\item Снижая давление, $ [x5~минут] $ будем снимать зависимость
	показаний манометра и вольтметра.
	\item Проведем измерения п. 3 при температуре
	$ T=50^\circ С~и~T=70^\circ С $. Результаты запишем в таблицу \ref{main_table}.
\end{enumerate}

\begin{table}[!h]\centering
	\begin{tabular}{cccc|cccc|cccc}
		\toprule
		dP, атм & U(P), мкВ & E    & dT   & dP, атм & U(P)   & E     & dT   & dP, атм & U(P)   & E    & dT          \\\midrule
		4,1     & 0,14      & 135  & 3,31 & 4       & 0,117  & 107,5 & 2,50 & 4       & 0,0971 & 87,1 & 1,95 \\
		3,51    & 0,121     & 116  & 2,84 & 3,5     & 0,1052 & 95,7  & 2,23 & 3,5     & 0,0872 & 77,2 & 1,73 \\
		2,7     & 0,0841    & 79,1 & 1,94 & 2,7     & 0,0781 & 68,6  & 1,59 & 2,7     & 0,0659 & 55,9 & 1,25 \\
		2       & 0,0573    & 52,3 & 1,28 & 2       & 0,0521 & 42,6  & 0,99 & 2       & 0,041  & 31   & 0,69 \\
		1       & 0,0261    & 21,1 & 0,52 & 1       & 0,0225 & 13    & 0,30 & 1       & 0,0189 & 8,9  & 0,2  \\
		\bottomrule      
	\end{tabular}
\caption{My caption}
\label{main_table}
\end{table}

\end{document}



















