\documentclass[a4paper,12pt]{article} % добавить leqno в [] для нумерации слева

%%% Работа с русским языком
\usepackage{cmap}					% поиск в PDF
\usepackage{mathtext}				% русские буквы в фомулах
\usepackage[T2A]{fontenc}			% кодировка
\usepackage[utf8]{inputenc}			% кодировка исходного текста
\usepackage[english,russian]{babel}	% локализация и переносы
\usepackage[left=2cm,right=2cm,top=2cm,bottom=2cm,footskip=1cm,includefoot]
{geometry}
%%% Дополнительная работа с математикой
\usepackage{amsmath,amsfonts,amssymb,amsthm,mathtools} %  AMS
\usepackage{icomma} % "Умная" запятая: $0,2$ --- число, $0, 2$ --- перечисление
%\usepackage{textcomp}
%% Номера формул
\mathtoolsset{showonlyrefs=true} % Показывать номера только у тех формул, на которые есть \eqref{} в тексте.

%% Перенос знаков в формулах (по Львовскому)
\newcommand*{\hm}[1]{#1\nobreak\discretionary{}
	{\hbox{$\mathsurround=0pt #1$}}{}}

%%% Работа с картинками
\usepackage{graphicx}  % Для вставки рисунков
\graphicspath{{images/}{images2/}}  % папки с картинками
\setlength\fboxsep{3pt} % Отступ рамки \fbox{} от рисунка
\setlength\fboxrule{1pt} % Толщина линий рамки \fbox{}
\usepackage{wrapfig} % Обтекание рисунков и таблиц текстом
%%% Работа с таблицами
\usepackage{array,tabularx,tabulary,booktabs} % Дополнительная работа с таблицами

\usepackage{longtable}  % Длинные таблицы
\usepackage{multirow} % Слияние строк в таблице
\newcommand{\RomanNumeralCaps}[1]
{\MakeUppercase{\romannumeral #1}}

\usepackage{pgfplots, pgfplotstable}
\pgfplotsset{compat=1.9}

\usepackage{circuitikz}

\begin{document}	
	
\begin{minipage}[l]{0.3\textwidth}
	\textit{Работу выполнил}\\
	Просвирин Кирилл, 712гр.\\\\
	\textit{под руководством}\\
	А.В. Гаврикова, к.ф.-м.н.
\end{minipage}
\hfill
\begin{minipage}[l]{0.22\textwidth}
	Маршрут \RomanNumeralCaps{9} \\\\
	19 февраля 2018~г.,\\
	26 февраля 2018~г.\\
\end{minipage}
\\[20pt]
\begin{center}
	\LARGE{\textbf{Лабораторная работа № 2.1.6}\\
	Эффект Джоуля-Томсона \\[20pt]}
\end{center}

\textbf{Цель работы: }определение изменения температуры углекислого газа
при протекании через малопроницаемую перегородку при различных начальных
значенях давления и температуры; вычисление по результатам опытов
коэффициентов Ван-дер-Ваальса.\\

\textbf{В работе используется: }Термостат, дифференциальная термопара,
микровольтметр, манометр, установка (баллон, труба и т.д.).


\section{Теоретическая справка}

Эффект Джоуля–Томсона — изменение температуры газа, медленно протекающего
из области высокого в область низкого давления в условиях хорошей тепловой
изоляции.\\

\noindent В условиях данной лабораторной работы газ проходит через пористую
перегородку из области с давлением $ P1 $ в область с атмосферным давлением 
$ P2 $. Рассмотрим стационарный поток между сечениями \RomanNumeralCaps{1}
и \RomanNumeralCaps{2} до и после перегородки. Считая стенки
адиабатическими и жёсткими, из закона сохранения энергии получим: 
\begin{equation}\label{}
	A_1-A_2=
	\left(U_2+\dfrac{\mu v_2^2}{2}\right)-
	\left(U_1+\dfrac{\mu v_1^2}{2}\right),
\end{equation}
где $ A_1=P_1V_1 $~--- работа совершенная над газом при прохождении
через \RomanNumeralCaps{1}, $ A_2=P_2V_2 $~--- работа совершенная над
газом при прохождении через \RomanNumeralCaps{2}. Тогда
\begin{equation}\label{}
	H_1-H_2=\dfrac{1}{2}\mu(v_2^2-v_1^2).
\end{equation}
Поскольку скорости по обе стороны от перегородки малы, энтальпию можно
считать неизменной. Тогда 
\begin{equation}\label{}
	\mu_{дт}=\dfrac{\Delta T}{\Delta P}=
	-\dfrac{(\partial H/\partial P)_T}{(\partial H/\partial T)_P}.
\end{equation}
Нетрудно показать, что для идеального газа коэффициент Джоуля–Томсона равен
нулю, а для газа Ван-дер-Ваальса 
\begin{equation}\label{}
	\mu_{дт}=\dfrac{\frac{2a}{RT}-b}{C_P}.
\end{equation}
Температура, при которой $ \mu_{дт} $ меняет знак, называют температурой
инверсии: 
\begin{equation}\label{}
	T_{инв}=\dfrac{2a}{Rb}
\end{equation}

\section{Экспериментальная установка}
\begin{figure}[!h]
	\begin{wrapfigure}[50]{l}{0.66\textwidth}\centering
		\includegraphics[width=0.7\textwidth]{scheme}
		\caption{Схема установки} 
	\end{wrapfigure}
	
	\textbf{Оборудование:}
	\begin{enumerate}
		\item Трубка
		\item Пористая перегородка
		\item Труба Дьюара
		\item Кольцо
		\item Змеевик
		\item Балластной балон
		\item Цифровой вольтметр
		\item 9.~~Спаи
		\setcounter{enumi}{9}
		\item Пробка
	\end{enumerate}
\end{figure}
\textbf{\\[1ex]Описание установки. }Рамка 1 жестко соединена с
проволокой 2, закрепленной вертикально в специальных зажимах 3,
позволяющих сообщить начальное закручивание для возбуждения крутильных
колебаний вокруг вертикальной оси. В рамке с помощью планки 4,
гаек 5 и винта 6 закрепляется твердое тело 7.

\section{Измерения}	 
\begin{table}[h]\centering\small
	\begin{tabular}{cccc|cccc|cccc}
	\toprule
	\multicolumn{4}{c|}{$ T=24^\circ С $} & 
	\multicolumn{4}{c|}{$ T=50^\circ С $} & 
	\multicolumn{4}{c}{$ T=70^\circ С $} \\\midrule
	$ \Delta P,~at $   & $ U,~\mu V $ & $ \varepsilon,~\mu V $    & $ \Delta T, ^\circ{С} $   
	& $ \Delta P,~at $   & $ U,~\mu  $ & $ \varepsilon,~\mu V $    & $ \Delta T, ^\circ{С} $ 
	& $ \Delta P,~at $   & $ U,~\mu  $ & $ \varepsilon,~\mu V $    & $ \Delta T, ^\circ{С} $           \\\midrule
	4,1     & 140      & 135  & 3,31   & 4       & 117  & 107,5 & 2,50 & 4       & 97 & 87,1 & 1,95 \\
	3,51    & 121     & 116  & 2,84   & 3,5     & 105 & 95,7  & 2,23 & 3,5     & 87 & 77,2 & 1,73 \\
	2,7     & 84    & 79,1 & 1,94   & 2,7     & 78 & 68,6  & 1,59 & 2,7     & 66 & 55,9 & 1,25 \\
	2       & 57    & 52,3 & 1,28   & 2       & 52 & 42,6  & 0,99 & 2       & 41 & 31  & 0,69 \\
	1       & 26    & 21,1 & 0,52   & 1       & 23 & 13    & 0,30 & 1       & 18 & 8,9  & 0,2  \\
	\bottomrule      
\end{tabular}

	\caption{Зависимость напряжения от перепада давлений}
	\label{main_table}
\end{table}
\noindent Для различных значений температур термостата будем снимать зависимость
показаний манометра и вольтметра. Данные измерений и их обработки 
приведены в таблице \ref{main_table}.

\begin{figure}[h!]\centering
		\begin{tikzpicture}
		\begin{axis}[ 
			scale=1.5, 
			xlabel={$\Delta P,~at$}, 
			ylabel={$\Delta T,~^\circ С$},
			domain = 0:5, xmin=0,ymin=0]
			
			\addplot[
				color=black, 
				only marks, 
				error bars/.cd, 
				y dir=both, 
				y explicit
			]
			table[
				col sep=comma, 
				x index=0, 
				y index=1, 
				y error index=2
			] {plot1.proc.csv};
		
			\addplot[
				color=gray, 
				only marks, 
				error bars/.cd,
				x dir=both, 
				x explicit, 
				y dir=both, 
				y explicit
			]
			table[
				col sep=comma, 
				x index=3, 
				y index=4, 
				y error index=5
			] {plot1.proc.csv};
			
			\addplot[
				color=blue, 
				only marks, 
				error bars/.cd,
				x dir=both, 
				x explicit, 
				y dir=both, 
				y explicit
			]
			table[
				col sep=comma, 
				x index=6, 
				y index=7, 
				y error index=8
			] {plot1.proc.csv};
			
			\addplot[color=darkgray] {0.9242*x-0.4822};
			\addplot[color=darkgray] {0.7537*x-0.4633};
			\addplot[color=darkgray] {0.6071*x-0.4337};
		\end{axis}
	\end{tikzpicture}
	\caption{График зависимости $ \Delta T(\Delta P) $}
\end{figure}

\section{Обработка}
Из данных эксперимента несложно рассчитать величины $ \mu_{ДТ} $:
\begin{align}
	\begin{array}{ccc}
		Эксперимент && Табличные данные\\
		\mu_{24}=(0,92\pm0,09)~{K}/{at} && 1,11~K/at\\ 
		\mu_{50}=(0,75\pm0,05)~{K}/{at} &~в~сравнении~с~ & 0,84~K/at\\ 
		\mu_{70}=(0,61\pm0,09)~{K}/{at} && 0,74~K/at\\ 
	\end{array}
\end{align}
Приведем формулы для подсчета коэффициентов и их погрешностей
\[
	a=\dfrac{RC_P}{2}\dfrac{\mu_1-\mu_2}{\frac{1}{T_1}-\frac{1}{T_2}},
	~~~
	b=C_p\dfrac{\mu_1T_1-\mu_2T_2}{T_2-T_1};
\]
\[
	\Delta a=\dfrac{RC_p}{2}
	\dfrac{\sqrt{\sigma_{\mu_1}^2+\sigma_{\mu_2}^2}}
	{\frac{1}{T_1}+\frac{1}{T_2}},
	~~~
	\Delta b=C_P=\dfrac{
		\sqrt{\sigma_{\mu_1}^2T_1^2+\sigma_{\mu_2}^2T_2^2}}{T_2-T_1}
\]
\[
	\Delta T_{инв}=T_{инв}
	\sqrt{\left(\dfrac{\Delta a}{a}\right)^2+
		\left(\dfrac{\Delta b}{b}\right)^2}
\]
Теперь можно найти коэффициенты Джоуля–Томсона для каждой серии измерений при различных температурах:
\begin{table}[h]	\centering
	\begin{tabular}{c|ccc}
		\toprule
		& $ [24;50]^\circ С $ 
		& $ [24;70]^\circ С $ 
		& $ [50;70]^\circ С $ \\
		\midrule
		$ a,~\frac{H\cdot м^4}{моль^2} $       
		& $ 1,49\pm0,8 $ & $ 1,66\pm0,5 $ & $ 1,92\pm0,7 $ \\[4pt]
		$ b,~\frac{м^3}{моль} $       
		& $ 346\pm237 $ & $ 418\pm130 $  & $ 511\pm279 $ \\[4pt]
		$ T_{инв},~K $       
		& $ 1053\pm925 $ & $ 975\pm419 $ & $ 922\pm621 $\\
		$ T_{кр},~K $       
		& $ 156\pm127 $ & $ 144\pm62 $ & $ 136\pm92 $\\
		\bottomrule
	\end{tabular}
	\caption{Коэффициенты Джоуля–Томсона}
	\label{my-label}
\end{table}
\section{Выводы}
Все порядки выдержаны, но в искомую $ \sigma $ ничего не попадает.
\end{document}
