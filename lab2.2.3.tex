\documentclass[a4paper,12pt]{article} % добавить leqno в [] для нумерации слева

%%% Работа с русским языком
\usepackage{cmap}					% поиск в PDF
\usepackage{mathtext}				% русские буквы в фомулах
\usepackage[T2A]{fontenc}			% кодировка
\usepackage[utf8]{inputenc}			% кодировка исходного текста
\usepackage[english,russian]{babel}	% локализация и переносы
\usepackage[left=2cm,right=2cm,top=2cm,bottom=2cm,footskip=1cm,includefoot]
{geometry}
%%% Дополнительная работа с математикой
\usepackage{amsmath,amsfonts,amssymb,amsthm,mathtools} % AMS
\usepackage{icomma} % "Умная" запятая: $0,2$ --- число, $0, 2$ --- перечисление
%\usepackage{textcomp}
%% Номера формул
%\mathtoolsset{showonlyrefs=true} % Показывать номера только у тех формул, на которые есть \eqref{} в тексте.

%% Перенос знаков в формулах (по Львовскому)
\newcommand*{\hm}[1]{#1\nobreak\discretionary{}
	{\hbox{$\mathsurround=0pt #1$}}{}}

%%% Работа с картинками
\usepackage{graphicx}  % Для вставки рисунков
\graphicspath{{images/}{images2/}}  % папки с картинками
\setlength\fboxsep{3pt} % Отступ рамки \fbox{} от рисунка
\setlength\fboxrule{1pt} % Толщина линий рамки \fbox{}
\usepackage{wrapfig} % Обтекание рисунков и таблиц текстом
%%% Работа с таблицами
\usepackage{array,tabularx,tabulary,booktabs} % Дополнительная работа с таблицами

\usepackage{longtable}  % Длинные таблицы
\usepackage{multirow} % Слияние строк в таблице
\newcommand{\RomanNumeralCaps}[1]
{\MakeUppercase{\romannumeral #1}}

\usepackage{pgfplots, pgfplotstable}
\pgfplotsset{compat=1.9}

\usepackage{circuitikz}

\begin{document}	
	
\begin{minipage}[l]{0.3\textwidth}
	\textit{Работу выполнил}\\
	Просвирин Кирилл, 712гр.\\\\
	\textit{под руководством}\\
	А.В. Гаврикова, к.ф.-м.н.
\end{minipage}
\hfill
\begin{minipage}[l]{0.2\textwidth}
	Маршрут \RomanNumeralCaps{9} \\\\
	5 февраля 2018~г.,\\
	12 февраля 2018~г.\\
\end{minipage}
\\[20pt]
\begin{center}
	\LARGE{\textbf{Лабораторная работа № 2.2.3}\\
		Определение теплопроводности газов при атмосферном давлении \\[20pt]}
\end{center}

\textbf{Цель работы: }определение теплопроводности воздуха или 
углекислого газа при атмосферном давлении и разных температурах по 
теплоотдаче нагреваемой током нити в цилиндрическом сосуде.\\

\textbf{В работе используется: }Водонагреватель; ММЭС Р4834; эталонное сопротивление 10 Ом; цифровой вольтметр B7-78/1; источник питания GPS-2303 GW INSTEK; 
термостат LT100.


\section{Постановка эксперимента}
\begin{wrapfigure}{r}{0.5\textwidth}\centering
	\includegraphics[width=0.5\textwidth]{schemee}
	\caption{Схема установки}
\end{wrapfigure}
Данная лабораторная работа предусматривает следующую методику измерений:
для разных значений температуры воды, протекающей через кожух, находим
зависимость падения напряжения $ U_n $ и $ U_0 $. После чего по этим
данным строим график зависимости $ Q $ от тока подогрева нити~$ I_H $.\\

\noindent\textbf{Обозначения:\\}
$ R_M $~--- магазин сопротивлений;\\
$ R_H $~--- сопротивление нити;\\
$ R_Э $~--- эталонное сопротивление.\\\\
Нетрудно показать, что если температура среды не зависит от времени, то 
теплопроводность $ \varkappa $ можно найти из соотношения
\begin{equation}\label{kappa}
	\varkappa=\dfrac{Q}{T_1-T_2}\dfrac{1}{2\pi L}\ln{\dfrac{r_2}{r_1}},
\end{equation}
где $ r_1 $~--- радиус нити, $ r_2 $~--- радиус цилиндра, 
$ L $~--- длина цилиндра, $ Q $~--- полный поток тепла,
$ \Delta T $~--- разность температур газа у поверхности нити и цилиндра.\\\\
Выделяемая мощность может быть найдена по формуле $ Q=10U_H/U_Э $
\newpage
\section{Проведение измерений}
\begin{enumerate}
	\item Снимем зависимость напряжения на нити $ U_H $ от напряжении
	на эталонном сопротивлении $ U_Э $.
\begin{table}[!h]\centering
	\begin{tabular}{ccccccccccc}
		\hline 
		&\multicolumn{2}{c}{$ T_1=297,8~K $}  
		&\multicolumn{2}{c}{$ T_2=304~K $}  
		&\multicolumn{2}{c}{$ T_3=312~K $}  
		&\multicolumn{2}{c}{$ T_4=322,8~K $}  
		&\multicolumn{2}{c}{$ T_5	=332,9~K $}  \\ 
		\hline 
		$№ $&$ U_0,~B $  &$ U_H,~B $  &$ U_0,~B $  &$ U_H,~B $  
		&$ U_0,~B $  &$ U_H,~B $  &$ U_0,~B $  &$ U_H,~B $
		&$ U_0,~B $  &$ U_H,~B $  \\
		\hline
		1&0,05  &0,755  
		&0,05  &0,762  
		&0,05  &0,766
		&0,05  &0,774
		&0,05  &0,782  \\ 
		2&0,075  &1,134  
		&0,075  &1,143  
		&0,075  &1,151
		&0,075  &1,162
		&0,075  &1,173  \\ 
		3&0,1  &1,513  
		&0,1  &1,524  
		&0,1  &1,536
		&0,1  &1,549
		&0,1  &1,566  \\ 
		4&0,125  &1,893
		&0,125  &1,907
		&0,125  &1,921
		&0,125  &1,938
		&0,125  &1,956  \\ 
		5&0,150  &2,273  
		&0,150  &2,289
		&0,150  &2,307
		&0,150  &2,328
		&0,150  &2,349  \\ 
		6&0,2  &3,037
		&0,2  &3,059
		&0,2  &3,082
		&0,2  &3,107
		&0,2  &3,137  \\
		7&0,225  &3,421
		&0,225  &3,446  
		&0,225  &3,471  
		&0,225  &3,501
		&0,225  &3,535  \\ 
		8&0,25  &2,273  
		&0,25  &3,848  
		&0,25  &3,862
		&0,25  &3,896
		&0,25  &3,932  \\ 
		\hline 
	\end{tabular} 
\caption{Основные данные измерений}
\end{table}
\item По данным из таблицы 1 построим график зависимости
выделяемой мощности от сопротивления и определим по нему угол наклона
$ dQ/dR $ и сопротивление нити $ R_0 $ при нулевой выделяемой
мощности.
\begin{figure}[!h]\centering\small
	\begin{tikzpicture}
	\pgfplotsset{width=15cm,
		compat=1.3,
		legend style={font=\footnotesize}}
	\begin{axis}[
	xlabel = {$R,~Ом$},
	ylabel = {$Q, 10^{-1}~Вт$},
	xmin = 15,
	ymin = 0,
	legend cell align=left,
	legend pos=north west]
	legend style={at={(0,0)},anchor=north east}
	
	\addlegendentry{$ T_1=298~K $}
	\addplot table[row sep=\\,
	y={create col/linear regression={y=Y}}] 
	%input table
	{
		X Y\\
		15.112	0.04\\
		15.123	0.09\\
		15.131	0.15\\
		15.141	0.24\\
		15.153	0.34\\
		15.186	0.61\\
		15.204	0.77\\
		15.230	0.95\\
	};
	\addlegendentry{$ T_2=304~K $}
	\addplot table[row sep=\\,
	y={create col/linear regression={y=Y}}] 
	{
		X Y\\
		15.230	0.04\\
		15.236	0.09\\
		15.244	0.15\\
		15.254	0.24\\
		15.264	0.34\\
		15.297	0.61\\
		15.315	0.78\\
		15.363	0.96\\
	};
	\addlegendentry{$ T_4=323~K $}
	\addplot table[row sep=\\,
	y={create col/linear regression={y=Y}}] 
	{
		X Y\\
		15.324	0.04\\
		15.348	0.09\\
		15.359	0.15\\
		15.3672	0.24\\
		15.3773	0.35\\
		15.4100	0.62\\
		15.4284	0.78\\
		15.4464	0.97\\
	};
	\addlegendentry{$ T_3=312~K $}
	\addplot table[row sep=\\,
	y={create col/linear regression={y=Y}}] 
	{
	X Y\\
	15.481	0.04\\
	15.487	0.09\\
	15.493	0.15\\
	15.502	0.24\\
	15.517	0.35\\
	15.543	0.62\\
	15.560	0.79\\
	15.586	0.97\\
	};
	\addlegendentry{$ T_5=333~K $}
	\addplot table[row sep=\\,
	y={create col/linear regression={y=Y}}] 
	{
		X Y\\
		15.632	0.04\\
		15.637	0.09\\
		15.659	0.16\\
		15.653	0.24\\
		15.662	0.35\\
		15.693	0.63\\
		15.711	0.80\\
		15.728	0.98\\
	};
	\end{axis}
	\end{tikzpicture}
	\caption{График зависимости сопротивления $ Q(R_H) $}
\end{figure}

\textbf{Замечание. }
\item Построим по значениям $ R_0 $ график зависимости сопротивления 
нити от температуры
\begin{table}[!h]\centering
	\begin{tabular}{cccccc}
		\hline 
		$ T,~K $& 297,8 & 304 & 312 & 322,8 & 332,9 \\ 
		\hline 
		$ R_0,~\Omega $& 15,11 & 15,21 & 15,33 & 15,48 & 15,63 \\ 
		\hline 
	\end{tabular} 
\caption{Значения $ R_0 $ при искомых температурах}
\end{table}

\begin{figure}[!h]\centering
	\begin{tikzpicture}
	\pgfplotsset{width=12cm,
		compat=1.3,
		legend style={font=\footnotesize}}
	\begin{axis}[
	xlabel = {$T,~K$},
	ylabel = {$R,~Ом$},
	legend cell align=left,
	legend pos=north west]
	legend style={at={(0,0)},anchor=north east}
	
	\addlegendentry{$ 0,0146x + 10,754 $}
	\addplot table[row sep=\\,
	y={create col/linear regression={y=Y}}] 
	%input table
	{
		X Y\\
		297.8	15.11\\
		304		15.21\\
		312		15.33\\
		322.8	15.48\\
		332.9	15.63\\
	};
	\end{axis}
	\end{tikzpicture}
	\caption{График зависимости сопротивления $ Q(R_H) $}
\end{figure}
Откуда находим угол наклона $ dR/dT=(0,0146\pm 0,0002)~Ом/K $.

\item Для каждой температуры прибора определим значение коэффициента
теплопроводности газа по формуле \eqref{kappa}.
\begin{table}[!h]\centering
	\begin{tabular}{cccccc}
		\hline 
		$ T,~K $ & 297,8 & 304 & 312 & 322,8 & 332,9\\
		\hline\\[-2ex]
		$ \dfrac{dQ}{dR},~Вт\cdot 10^{-2}/K $ 
		& 80,0 & 87,3 & 80,6 & 90,9 & 99,2\\[2ex] 
		\hline \\[-2ex]
		$ \sigma\left(\dfrac{dQ}{dR}\right),~Вт\cdot 10^{-2}/K $ 
		& 1,5 & 7,1 & 5 & 1,9 & 6,1\\[2ex]
		\hline\\[-2ex]
		$ \varkappa,~Дж\cdot 10^{-2}/K\cdot м $
		& 2,68 & 2,93 & 2,70 & 3,1 & 3,3\\ 
		\hline\\[-2ex]
		$ \sigma\varkappa,~Дж\cdot 10^{-2}/K\cdot м $
		& 0,06 & 0,24 & 0,07 & 0,2 & 0,2\\ 
		\hline 
	\end{tabular} \caption{Расчет коэффициента теплопроводности}
\end{table}

\item По данным таблицы 3 построим график зависимости теплопроводности
от температуры. И в предположении, что $ \varkappa=AT^\beta $, найдем 
показатель степени $ \beta $.

\pgfplotstableread{
	5.6964	-3.6189
	5.7170	-3.5315
	5.7430	-3.6111
	5.7770	-3.4908
	5.8078	-3.4040
}\datatable

\begin{figure}[!h]\centering
	\pgfplotstableread{
		X			Y
		5.69642212	-3.618994248
		5.717027701	-3.531566343
		5.743003188	-3.611151071
		5.777032936	-3.49083597
		5.807842145	-3.404007861
	}\datatable
	
	\begin{tikzpicture}
	\begin{axis}[
	xlabel = {$\ln T,~K$},
	ylabel = {$\ln\varkappa,~~Дж/K\cdot м $},
	width = 0.7\textwidth,
	height = 0.45\textheight,
	legend style={at={(0.3,0.98)},anchor=north east}]
	
	
	\addplot [only marks, mark = *] table {\datatable};
	\addlegendentry{$ 1,70x + 13,31 $}
	\addplot [thick, blue] table[
	y={create col/linear regression={y=Y}}
	] % compute a linear regression from the input table
	{\datatable};
	\end{axis}
	\end{tikzpicture}
\end{figure}



\end{enumerate}

Из наклона прямой на графике находим $ \beta=(1,70\pm 0,16) $
\section{Итоги}

Полученные в результате настоящей лабораторной
работы значения коэффициента теплопроводности совпадают с табличными
значениями в пределах погрешности, а графики имеют <<хороший>>, линейный 
вид $ (\mathfrak{R}\approx0,9775). $\\\\
\textbf{Комментарий. }Погрешности измерений были 
подсчитаны в среде \textit{<<OriginPro>>}.



\end{document}




Длина трубки Т & $L = 64~\cm$ & $\pm 1~\cm$\\
Поперечное сечение резервуара II &
$\zeta^2 = (19.5 \times 19.5)~\cm$ & \quad кажд. $\pm 0.2~\cm$ \\
Эффективная высота резервуара II\quad & $h = 31.5~\cm$ & $\pm 0.2~\cm$ \\
