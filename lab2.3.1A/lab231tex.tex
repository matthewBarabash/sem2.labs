\documentclass[a4paper,12pt]{article} % добавить leqno в [] для нумерации слева

%%% Работа с русским языком
\usepackage{cmap}					% поиск в PDF
\usepackage{mathtext}				% русские буквы в фомулах
\usepackage[T2A]{fontenc}			% кодировка
\usepackage[utf8]{inputenc}			% кодировка исходного текста
\usepackage[english,russian]{babel}	% локализация и переносы
\usepackage[left=2cm,right=2cm,top=2cm,bottom=2cm,footskip=1cm,includefoot]
{geometry}
%%% Дополнительная работа с математикой
\usepackage{amsmath,amsfonts,amssymb,amsthm,mathtools} % AMS
\usepackage{icomma} % "Умная" запятая: $0,2$ --- число, $0, 2$ --- перечисление
%\usepackage{textcomp}
%% Номера формул
%\mathtoolsset{showonlyrefs=true} % Показывать номера только у тех формул, на которые есть \eqref{} в тексте.

%% Перенос знаков в формулах (по Львовскому)
\newcommand*{\hm}[1]{#1\nobreak\discretionary{}
	{\hbox{$\mathsurround=0pt #1$}}{}}

%%% Работа с картинками
\usepackage{graphicx}  % Для вставки рисунков
\graphicspath{{images/}{images2/}}  % папки с картинками
\setlength\fboxsep{3pt} % Отступ рамки \fbox{} от рисунка
\setlength\fboxrule{1pt} % Толщина линий рамки \fbox{}
\usepackage{wrapfig} % Обтекание рисунков и таблиц текстом
%%% Работа с таблицами
\usepackage{array,tabularx,tabulary,booktabs} % Дополнительная работа с таблицами

\usepackage{longtable}  % Длинные таблицы
\usepackage{multirow} % Слияние строк в таблице
\newcommand{\RomanNumeralCaps}[1]
{\MakeUppercase{\romannumeral #1}}

\usepackage{pgfplots, pgfplotstable}
\pgfplotsset{compat=1.9}

\usepackage{circuitikz}

\begin{document}	
	
\begin{minipage}[l]{0.3\textwidth}
	\textit{Работу выполнил}\\
	Просвирин Кирилл, 712гр.\\\\
	\textit{под руководством}\\
	А.В. Гаврикова, к.ф.-м.н.
\end{minipage}
\hfill
\begin{minipage}[l]{0.2\textwidth}
	Маршрут \RomanNumeralCaps{9} \\\\
	12 апреля 2018~г.,\\
	18 фпреля 2018~г.\\
\end{minipage}
\\[20pt]
\begin{center}
	\LARGE{\textbf{Лабораторная работа № 2.3.1Ф}\\
		Получение и измерение вакуума \\[20pt]}
\end{center}

\textbf{Цель работы: }1) измерение объемов форвакуумной и 
высоковакуумной частей установки ; 2) определение скорости
откачки системы в стационарном режиме, а также по
ухудшению и по улучшению вакуума.\\

\textbf{В работе используется: }
Компактный безмасляный высоковакуумный Pfeiffer Vacuum серии
HiCube 80 Eco, вакууметр Pfeiffer Vacuum серии DigiLine и
вакуумные компоненты типов ISO-K, ISO-F, ISO-KF. .


\section{Экспериментальная установка}
\begin{figure}[!h]
	\begin{wrapfigure}[50]{l}{0.5\textwidth}
		\vspace{-4ex}
		\includegraphics[width=0.45\textwidth]{scheme}
		\caption{Схема установки} 
	\end{wrapfigure}
	
	\textbf{Оборудование:}
	\begin{enumerate}
		\item БУ — блок управления
		\item ДН
		— диафрагменный насос
		\item ТМН — турбомолекулярный насос
		\item К
		— вакуумная камера
		\item ШЗ
		— шиберный затвор
		\item МК1-3 — мембранные краны
		\item В1
		— терморезистоный вакууметр
		\item В2
		— комбинированный вакууметр
		\item КН
		— кран-натекатель
		\item З
		— заглушка
		\item Д
		— диафрагма
	\end{enumerate}
\end{figure}













\end{document}























